\documentclass[11pt]{beamer}
\usetheme{Copenhagen}
\usepackage[utf8]{inputenc}
\usepackage{hyperref}
\usepackage{amsmath}
\usepackage{amsfonts}
\usepackage{amssymb}
\usepackage{graphicx}
\usepackage{xcolor}
\usepackage{tikz}
\usetikzlibrary{shapes, arrows, calc, arrows.meta, fit, positioning} % these are the parameters passed to the library to create the node graphs  
\tikzset{  
    -Latex,auto,node distance =1.5 cm and 1.3 cm, thick,% node distance is the distance between one node to other, where 1.5cm is the length of the edge between the nodes  
    state/.style ={ellipse, draw, minimum width = 0.9 cm}, % the minimum width is the width of the ellipse, which is the size of the shape of vertex in the node graph  
    point/.style = {circle, draw, inner sep=0.18cm, fill, node contents={}},  
    bidirected/.style={Latex-Latex,dashed}, % it is the edge having two directions  
    el/.style = {inner sep=2.5pt, align=right, sloped}  
}  

\usepackage[backend=bibtex,style=ieee]{biblatex}
\addbibresource{group_protocols_auth.bib}

\author{Andrei Cristian\\
\href{mailto:andrei.cristian1@info.uaic.ro}{andrei.cristian1@info.uaic.ro}}
\title{Verificarea protocoalelor de autentificare de grup prin Scyther}
\setbeamercovered{transparent} 
\setbeamertemplate{navigation symbols}{} 
%\logo{} 
%\institute{} 
\date{\today} 
%\subject{} 
\begin{document}

\begin{frame}
\titlepage
\end{frame}

\begin{frame}
\tableofcontents
\end{frame}

\section{Introducere}

\begin{frame}{Introducere}

Lucrarea "Verifying Group Authentication Protocols by Scyther", Huihui Yang, Vladimir Oleshchuk, and Andreas Prinz(University of Agder, Kristiansand, Norway) prezinta analiza a doua protocoale complexe de autentificare de grup folosind Scyther.

Din cauza limitarii utilitarului, doar un subset de proprietati de securitate au fost verificate:
\begin{itemize}
\item autentificare mutuala;
\item autentificare cu cheie implicita \footnote{proprietatea in care una dintre parti este asigurata ca nicio alta parte in afara de o a doua parte identificata in mod specific nu poate avea acces la o anumita cheie secreta};
\item siguranta impotriva atacurilor de impersonare si adversarilor pasivi.
\end{itemize}

\end{frame}

\begin{frame}{Scyther}
\begin {itemize}
\item Pentru verificarea securitatii protocoalelor exista doua abordari principale: securitatea demonstrabila (eng. \textit{provable security}) si metodele formale (eng. \textit{formal methods}).
\item \textbf{Scyther} este un utilitar de verificare formala si este conceput pentru verificarea automata a protocoalelor de securitate.
\item Modelul adversarial este predefinit si anume modelul Dolev-Yao. Aceasta abordare simplifica formalizarea protocoalelor de securitate si il face mai usor de folosit pentru utilizatorii noi.
\item Poate oferi clase de comportament de protocol spre deosebire de doar urmele de atac (furnizate in cazul altor utilitare).

\end{itemize}
\end{frame}


\begin{frame}[t,allowframebreaks]{Bibliografie}
\printbibliography
\end{frame}

\end{document}